%Advancments%
The evolution of AI in gaming, particularly through the development 
of AlphaGo, AlphaGo Zero, and MuZero, highlights remarkable advancements
in reinforcement learning and artificial intelligence. AlphaGo, the pioneering model, 
combined supervised learning and reinforcement learning to master the 
complex game of Go, setting the stage for AI to exceed human capabilities 
in well-defined strategic games. Building on, AlphaGo 
Zero eliminated the reliance on human data, introducing a fully 
self-supervised approach that demonstrated greater efficiency and 
performance by learning solely through self-play. MuZero took this 
innovation further by generalizing beyond specific games like Go, 
Chess, and Shogi, employing model-based reinforcement learning to 
predict dynamics without explicitly knowing the rules of the environment.
Completing on these three models, here are some of the advancements that 
developed from them: AlphaZero and MiniZero; and one of the most used 
in generating AI models, Multi-agent models. 
\subsection*{AlphaZero}
While AlphaGo Zero was an impressive feat, designed specifically to 
master the ancient game of Go through self-play, AlphaZero developes it 
by generalizing its learning framework to 
include multiple complex games: chess, shogi (Japanese chess), and Go. 
the key advancement is in its ability to apply the same algorithm 
across different games without requiring game-specific adjustments. 
AlphaZero's neural network is trained through self-play, predicting 
the move probabilities and game outcomes for various positions. This 
prediction is then used to guide the MCTS, which explores potential 
future moves and outcomes to determine the best action. Through 
iterative self-play and continuous refinement of the neural network, 
AlphaZero efficiently learns and improves its strategies across 
different games\cite{AD3}.
Another significant improvement is in AlphaZero’s generalized algorithm, 
is that it does not need to be fine-tuned for each specific game. This was 
a departure from AlphaGo Zero’s Go-specific architecture, making 
AlphaZero a more versatile AI system.\\ AlphaZero's architecture integrates a 
single neural network that evaluates both the best moves and the 
likelihood of winning from any given position, streamlining the 
learning process by eliminating the need for separate policy and 
value networks used in earlier systems. This innovation not only
enhances computational efficiency but also enables AlphaZero to adopt 
unconventional and creative playing styles that diverge from 
established human strategies.
\subsection*{MiniZero}
MiniZero is a a zero-knowledge learning framework that supports four 
state-of-the-art algorithms, including AlphaZero, MuZero, Gumbel 
AlphaZero, and Gumbel MuZero\cite{AD1}. Gumbel AlphaZero and Gumbel 
MuZero are variants of the AlphaZero and MuZero algorithms that 
incorporate Gumbel noise into their decision-making process to improve 
exploration and planning efficiency in reinforcement learning tasks.
Gumbel noise is a type of stochastic noise sampled from the Gumbel distribution, 
commonly used in decision-making and optimization problems.\\
MiniZero is a simplified version of the original MuZero algorithm,
which is designed to be have a more simplified architecture reducing 
the complexity of the neural network used to model environment dynamics, 
making it easier to implement and experiment with. This simplification 
allows MiniZero to perform well in smaller environments with fewer states 
and actions, offering faster training times and requiring fewer computational 
power compared to MuZero. 
\subsection*{Multi-agent models}
Multi-agent models in reinforcement learning (MARL) represent an extension of traditional single-agent reinforcement learning. 
In these models, multiple agents are simultaneously 
interacting, either competitively or cooperatively, making decisions 
that impact both their own outcomes and those of other agents. The complexity 
in multi-agent systems arises from the dynamic nature of the environment, 
where the actions of each agent can alter the environment and the states 
of other agents. Unlike in single-agent environments, where the agent 
learns by interacting with a static world, multi-agent systems require 
agents to learn not only from their direct experiences but also from the 
behaviors of other agents, leading to a more complex learning process. 
Agents must adapt their strategies based on what they perceive other agents 
are doing, and this leads to problems such as strategic coordination, 
deception, negotiation, and competitive dynamics. In competitive scenarios, 
agents might attempt to outwit one another, while in cooperative scenarios, 
they must synchronize their actions to achieve a common goal\cite{AD2}.\\
AlphaGo and AlphaGo Zero are not designed to handle 
multi-agent environments. The core reason lies in their foundational 
design, which assumes a single agent interacting with a static 
environment. AlphaGo and AlphaGo Zero both rely on model-based reinforcement 
learning and self-play, where a single agent learns by interacting 
with itself or a fixed opponent, refining its strategy over time. 
However, these models are not built to adapt to the dynamic nature of 
multi-agent environments, where the state of the world constantly 
changes due to the actions of other agents. In AlphaGo and AlphaGo Zero, 
the environment is well-defined, and the agent’s objective is to 
optimize its moves based on a fixed set of rules. The agents in these 
models do not need to account for the actions of other agents in 
real-time or consider competing strategies, which are essential in 
multi-agent systems. Additionally, AlphaGo and AlphaGo Zero are not 
designed to handle cooperation or negotiation, which are 
key aspects of multi-agent environments.\\
On the other hand, MuZero offers a more flexible
framework that can be adapted to multi-agent environments. Unlike 
AlphaGo and AlphaGo Zero, MuZero operates by learning the dynamics of 
the environment through its interactions, rather than relying on a 
fixed model of the world. This approach allows MuZero to 
adapt to various types of environments, whether single-agent or 
multi-agent, by learning to predict the consequences of actions 
without needing explicit knowledge of the environment’s rules. 
The key advantage of MuZero in multi-agent settings is its ability 
to plan and make decisions without needing to model the entire system 
upfront. In multi-agent environments, this ability becomes essential, 
as MuZero can dynamically adjust its strategy based on the observed 
behavior of other agents. By learning not just the immediate outcomes 
but also the strategic implications of others' actions, MuZero can navigate both competitive and cooperative settings.