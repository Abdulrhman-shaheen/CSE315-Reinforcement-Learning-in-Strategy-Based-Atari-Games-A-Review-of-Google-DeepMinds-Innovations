%BACKGROUND SECTION%
Reinforcement Learning (RL) is a key area of machine learning that focuses on
learning through interaction with the environment. In RL, an agent takes
actions (A) in specific states (S) with the goal of maximizing the rewards (R)
received from the environment. The foundations of RL can be traced back to
1911, when Thorndike introduced the Law of Effect, suggesting that actions
leading to favorable outcomes are more likely to be repeated, while those
causing discomfort are less likely to recur \cite{bg1}.\\ RL emulates the human
learning process of trial and error. The agent receives positive rewards for
beneficial actions and negative rewards for detrimental ones, enabling it to
refine its policy function—a strategy that dictates the best action to take in
each state. That's said, for a give agent in state $u$, if it takes action $u$,
then the immediate reward $r$ can be modeled as $r(x, u) = \mathbb{E}[r_t \mid
    x=x_{t-1}, u=u_{t-1}]$.\\ So for a full episode of $T$ steps, the cumulative
reward $R$ can be modeled as $R = \sum_{t=1}^{T} r_t$.\\
\subsection{Markov Decision Process (MDP)}

In reinforcement learning, the environment is often modeled as a \textbf{Markov
    Decision Process (MDP)}, which is defined as a tuple $(S, A, P, R, \gamma)$,
where:
\begin{itemize}
    \item \( S \) is the set of states,
    \item \( A \) is the set of actions,
    \item \( P \) is the transition probability function,
    \item \( R \) is the reward function, and
    \item \( \gamma \) is the discount factor.
\end{itemize}

The MDP framework is grounded in \textbf{sequential decision-making}, where the
agent makes decisions at each time step based on its current state. This
process adheres to the \textbf{Markov property}, which asserts that the future
state and reward depend only on the present state and action, not on the
history of past states and actions.

Formally, the Markov property is represented by:

\[
    P(s'\mid s, a) = \mathbb{P}[s_{t+1} = s' \mid s_t = s, a_t = a]
\]

which denotes the probability of transitioning from state $s$ to state $s'$
when action $a$ is taken.

The reward function \( R \) is similarly defined as:

\[
    R(s, a) = \mathbb{E}[r_t \mid s_{t-1} = s, a_{t-1} = a]
\]

which represents the expected reward received after taking action $a$ in state
$S$.
\subsection{Policy and Value Functions}
In reinforcement learning, an agent's goal is to find the optimal policy that
the agent should follow to maximize cumulative rewards over time. To facilitate
this process, we need to quantify the desirability of a given state, which is
done through the \textbf{value function} $V(s)$. Value function estimates the
expected cumulative reward an agent will receive starting from state \( s \)
and continuing thereafter. In essence, the value function reflects how
beneficial it is to be in a particular state, guiding the agent's
decision-making process. The value function is defined as:
\[
    V(s) = \mathbb{E}[G_t \mid s_t = s]
\]
where \( G_t \) is the cumulative reward from time step $t$ onwards. From here
we can define the \textbf{action-value function} under policy $\pi$ $Q_\pi(s,
    a)$, which again, estimates the expected cumulative reward from the state $s$
and taking action $a$ and then following policy $\pi$:
\[
    Q_\pi(s, a) = \mathbb{E_\pi}[G_t \mid s_t = s, a_t = a]
\]
\[
    = \mathbb{E_\pi}[r_t + \gamma r_{t+1} + \gamma^2 r_{t+2} + \ldots \mid s_t = s, a_t = a]
\]

where $\gamma$ is the discount factor, which is a decimal value between 0 and 1
that detemines how much we care about immediate rewards versus future reward
rewards \cite{bg2}.\\

\subsection{Reinforcement Learning Algorithms}
There are multiple reinforcement learning algorithms that have been developed
that falls under a lot of categories. But, for the sake of this review, we will
focus on the following algorithms that have been used by the Google DeepMind
team in their reinforcement learning models.\\

\subsubsection{\textbf{Monte Carlo Algorithm}}
The Monte Carlo Algorithm is a model-free reinforcement learning algorithm used 
to estimate the value of states or state-action pairs under a given policy by averaging
 the returns of multiple episodes. This method alternates between two main steps: policy
  evaluation and policy improvement.

\begin{itemize}
    \item \textbf{Policy Evaluation:} The algorithm evaluates the value of a state or state-action pair by averaging the returns from multiple episodes that include that state or state-action pair.
    \item \textbf{Policy Improvement:} The algorithm improves the policy by selecting the action that maximizes the value of the state-action pair.
\end{itemize}

Since the algorithm is model-free and does not assume any prior knowledge of
the environment's dynamics, it focuses on estimating state-action pair values
(\( Q(s, a) \)) rather than just state values. The Monte Carlo Algorithm
typically follows these steps:

\begin{enumerate}
    \item Initialize the state-action pair values \( Q(s, a) \) arbitrarily.
    \item Start at a random state and take a random action, ensuring the possibility of
          visiting all state-action pairs over time.
    \item Follow the current policy \( \pi \) until a terminal state is reached.
    \item Update the value of each state-action pair \( Q(s, a) \) encountered in the
          episode using the observed returns.
    \item For each state visited in the episode, update the policy \( \pi(s) \) to select
          the action that maximizes \( Q(s, a) \):
          \[
              \pi(s) = \arg\max_a Q(s, a).
          \]
\end{enumerate}

This algorithm is particularly well-suited for environments that are
\emph{episodic}, where each episode ends in a terminal state after a finite
number of steps.

