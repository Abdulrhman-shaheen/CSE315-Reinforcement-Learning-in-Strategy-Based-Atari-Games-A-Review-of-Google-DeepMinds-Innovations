\documentclass[conference]{IEEEtran}
\IEEEoverridecommandlockouts
% The preceding line is only needed to identify funding in the first footnote. If that is unneeded, please comment it out.
\usepackage{cite}
\usepackage{amsmath,amssymb,amsfonts}
\usepackage{algorithmic}
\usepackage{graphicx}
\usepackage{textcomp}
\usepackage{xcolor}
\usepackage{tabularx}
\usepackage{url}
\usepackage{titlesec}
\def\BibTeX{{\rm B\kern-.05em{\sc i\kern-.025em b}\kern-.08em
    T\kern-.1667em\lower.7ex\hbox{E}\kern-.125emX}}
\begin{document}

\title{Reinforcement Learning in Strategy-Based and Atari Games: A Review of Google DeepMind's Innovations\\}

\author{
    \IEEEauthorblockN{Abdelrhman Shaheen\IEEEauthorrefmark{1},
                      Anas Badr\IEEEauthorrefmark{1},
                      Ali Abohendy\IEEEauthorrefmark{1},
                      Hatem Alsaadawy\IEEEauthorrefmark{1},
                      Nadine Alsayad\IEEEauthorrefmark{1},
                      Ehab H. El-Shazly\IEEEauthorrefmark{1}\IEEEauthorrefmark{2}}
    \IEEEauthorblockA{\IEEEauthorrefmark{1}Egypt Japan University of Science and Technology (E-JUST), Alexandria, Egypt \\
    \{abdelrhman.shaheen, anas.badr, ali.abohendy, hatem.alsaadawy, nadine.alsayad, ehab.elshazly\}@ejust.edu.eg}
    \IEEEauthorblockA{\\ \IEEEauthorrefmark{2}Radiation Engineering Department, The National Center for Radiation Research and Technology,\\
Egyptian Atomic Energy Authority, Egypt}
}


\maketitle
\thispagestyle{plain}
\pagestyle{plain}
\begin{abstract}

    \input{sections/1Abstract/abstract.tex}

\end{abstract}

\begin{IEEEkeywords}
    Deep Reinforcement Learning, Google DeepMind, AlphaGo, AlphaGo Zero, MuZero, Atari Games, Go, Chess, Shogi,
\end{IEEEkeywords}

\section{Introduction}

Artificial Intelligence (AI) has revolutionized the gaming industry, both as a
tool for creating intelligent in-game opponents and as a testing environment
for advancing AI research. Games serve as an ideal environment for training and
evaluating AI systems because they provide well-defined rules, measurable
objectives, and diverse challenges \cite{skinner1938}\cite{georgios2018}. 
From simple puzzles to complex strategy games, AI research in gaming has pushed 
the boundaries of machine learning and reinforcement learning 
\cite{samuel1959}\cite{minsky1961}\cite{kaelbling1996}\cite{suttonbarto2018}. 
Also, the benefits from such employment helped game developers to realize 
the power of AI methods to analyze large volumes of player data and optimize 
game designs \cite{georgios2018}. \\

Atari games, in particular, with their retro visuals and straightforward
mechanics, offer a challenging yet accessible benchmark for testing AI
algorithms. The simplicity of Atari games hides complexity: they require 
strategies that involve planning, adaptability, and fast decision-making,
making them a good testing environment for evaluating AI’s ability to learn 
and generalize \cite{bellemare2013}\cite{mnih2015}. The development of AI in games 
has been a long journey, starting with rule-based systems and evolving into 
more sophisticated machine learning models \cite{samuel1959}\cite{minsky1961}. 
However, early machine learning models faced challenges in decision-making 
within interactive game environments \cite{puterman1994}\cite{howard1960}. 
Traditional supervised and unsupervised models depended on static datasets 
without the ability to interact dynamically with the environment. 
To overcome this problem, game developers and researchers began to employ 
reinforcement learning (RL) \cite{bellman1957}\cite{barto1983}\cite{sutton1988}\cite{watkins1992}\cite{dayan1992}. 
Years later, deep learning (DL) showed remarkable success in computer vision 
and video games \cite{lecun2015}\cite{goodfellow2016}\cite{I2}. The combination of RL and DL 
resulted in Deep Reinforcement Learning (DRL), enabling agents to learn directly 
from high-dimensional sensory data \cite{tsitsiklis1997}\cite{williams1992}. 
The first notable use of DRL was in Atari games \cite{I3}\cite{mnih2015}. \\

One of the leading companies in applying DRL to games is Google DeepMind. 
This company is widely recognized for developing advanced AI models, including 
those for games. Prior to AlphaGo, DeepMind had already contributed 
substantially to DRL through Atari benchmarks and novel algorithmic 
developments \cite{mnih2015}\cite{bellemare2013}. For sequential decision-making, 
DeepMind combined RL techniques with neural networks to create models capable 
of memory and reasoning, such as Neural Turing Machines (NTMs) \cite{I4}. 
They then introduced the Deep Q-Network (DQN) algorithm, which combined 
Q-learning \cite{watkins1992}\cite{watkins1989} with deep neural networks, allowing agents to 
approximate the Q-function from high-dimensional inputs \cite{I6}. 
The DQN represented a major breakthrough in deep RL, as it was the first 
algorithm to learn directly from raw pixels and achieve human-level performance 
on Atari games \cite{mnih2015}. \\

To further enhance learning efficiency, DeepMind introduced experience replay 
\cite{I7}\cite{lin1992}, which stabilized training by breaking correlations in sequential data. 
They then developed asynchronous methods such as A3C, which improved stability 
and scalability in DRL \cite{I8}\cite{konda2000}\cite{tsitsiklis1997}. These developments 
enabled DeepMind to build AlphaGo, the first AI model to defeat the world 
champion in the game of Go, a historic milestone in both AI and reinforcement 
learning \cite{Silver2016}. \\ 

Our paper is similar to Shao et al. \cite{I12}, as we discussed the 
developments that Google DeepMind made in developing AI models for 
games and the advancements that they made over the last years to develop 
the models and the future directions of implementing DRL in games; 
how this implementation helps in developing real life applications. The
main contribution in our paper is the comprehensive details of the four 
models AlphaGo, AlphaGo Zero, AlphaZero, and MuZero, focusing on the key innovations for 
each model, how the training process was done, challenges that each model
faced and the improvements that were made, and the performance benchmarks. 
Studying each one of these models in detail helps in understanding how 
RL was developed in games reaching the current state, by which it is 
now used in real life applications. Also we discussed the advancements 
in these four AI models, reaching to the future directions.


\section{Background}
%BACKGROUND SECTION%
Reinforcement Learning (RL) is a key area of machine learning that focuses on
learning through interaction with the environment. In RL, an agent takes
actions (A) in specific states (S) with the goal of maximizing the rewards (R)
received from the environment. The foundations of RL can be traced back to
1911, when Thorndike introduced the Law of Effect, suggesting that actions
leading to favorable outcomes are more likely to be repeated, while those
causing discomfort are less likely to recur \cite{bg1}.\\ RL emulates the human
learning process of trial and error. The agent receives positive rewards for
beneficial actions and negative rewards for detrimental ones, enabling it to
refine its policy function—a strategy that dictates the best action to take in
each state. That's said, for a give agent in state $u$, if it takes action $u$,
then the immediate reward $r$ can be modeled as $r(x, u) = \mathbb{E}[r_t \mid
    x=x_{t-1}, u=u_{t-1}]$.\\ So for a full episode of $T$ steps, the cumulative
reward $R$ can be modeled as $R = \sum_{t=1}^{T} r_t$.\\
\subsection{Markov Decision Process (MDP)}

In reinforcement learning, the environment is often modeled as a \textbf{Markov
    Decision Process (MDP)}, which is defined as a tuple $(S, A, P, R, \gamma)$,
where:
\begin{itemize}
    \item \( S \) is the set of states,
    \item \( A \) is the set of actions,
    \item \( P \) is the transition probability function,
    \item \( R \) is the reward function, and
    \item \( \gamma \) is the discount factor.
\end{itemize}

The MDP framework is grounded in \textbf{sequential decision-making}, where the
agent makes decisions at each time step based on its current state. This
process adheres to the \textbf{Markov property}, which asserts that the future
state and reward depend only on the present state and action, not on the
history of past states and actions.

Formally, the Markov property is represented by:

\[
    P(s'\mid s, a) = \mathbb{P}[s_{t+1} = s' \mid s_t = s, a_t = a]
\]

which denotes the probability of transitioning from state $s$ to state $s'$
when action $a$ is taken.

The reward function \( R \) is similarly defined as:

\[
    R(s, a) = \mathbb{E}[r_t \mid s_{t-1} = s, a_{t-1} = a]
\]

which represents the expected reward received after taking action $a$ in state
$S$.
\subsection{Policy and Value Functions}
In reinforcement learning, an agent's goal is to find the optimal policy that
the agent should follow to maximize cumulative rewards over time. To facilitate
this process, we need to quantify the desirability of a given state, which is
done through the \textbf{value function} $V(s)$. Value function estimates the
expected cumulative reward an agent will receive starting from state \( s \)
and continuing thereafter. In essence, the value function reflects how
beneficial it is to be in a particular state, guiding the agent's
decision-making process. The value function is defined as:
\[
    V(s) = \mathbb{E}[G_t \mid s_t = s]
\]
where \( G_t \) is the cumulative reward from time step $t$ onwards. From here
we can define the \textbf{action-value function} under policy $\pi$ $Q_\pi(s,
    a)$, which again, estimates the expected cumulative reward from the state $s$
and taking action $a$ and then following policy $\pi$:
\[
    Q_\pi(s, a) = \mathbb{E_\pi}[G_t \mid s_t = s, a_t = a]
\]
\[
    = \mathbb{E_\pi}[r_t + \gamma r_{t+1} + \gamma^2 r_{t+2} + \ldots \mid s_t = s, a_t = a]
\]

where $\gamma$ is the discount factor, which is a decimal value between 0 and 1
that detemines how much we care about immediate rewards versus future reward
rewards \cite{bg2}.\\

\subsection{Reinforcement Learning Algorithms}
There are multiple reinforcement learning algorithms that have been developed
that falls under a lot of categories. But, for the sake of this review, we will
focus on the following algorithms that have been used by the Google DeepMind
team in their reinforcement learning models.\\

\subsubsection{\textbf{Monte Carlo Algorithm}}
The Monte Carlo Algorithm is a model-free reinforcement learning algorithm used 
to estimate the value of states or state-action pairs under a given policy by averaging
 the returns of multiple episodes. This method alternates between two main steps: policy
  evaluation and policy improvement.

\begin{itemize}
    \item \textbf{Policy Evaluation:} The algorithm evaluates the value of a state or state-action pair by averaging the returns from multiple episodes that include that state or state-action pair.
    \item \textbf{Policy Improvement:} The algorithm improves the policy by selecting the action that maximizes the value of the state-action pair.
\end{itemize}

Since the algorithm is model-free and does not assume any prior knowledge of
the environment's dynamics, it focuses on estimating state-action pair values
(\( Q(s, a) \)) rather than just state values. The Monte Carlo Algorithm
typically follows these steps:

\begin{enumerate}
    \item Initialize the state-action pair values \( Q(s, a) \) arbitrarily.
    \item Start at a random state and take a random action, ensuring the possibility of
          visiting all state-action pairs over time.
    \item Follow the current policy \( \pi \) until a terminal state is reached.
    \item Update the value of each state-action pair \( Q(s, a) \) encountered in the
          episode using the observed returns.
    \item For each state visited in the episode, update the policy \( \pi(s) \) to select
          the action that maximizes \( Q(s, a) \):
          \[
              \pi(s) = \arg\max_a Q(s, a).
          \]
\end{enumerate}

This algorithm is particularly well-suited for environments that are
\emph{episodic}, where each episode ends in a terminal state after a finite
number of steps.



\section{Classification of Methods}
In this section, we will be discussing the four models of Google DeepMind:
\subsection{AlphaGo}
\input{sections/4AlphaGo/AlphaGo.tex}

\subsection{AlphaGo Zero}
\input{sections/5AlphaGo Zero/AlphaGoZero.tex}

\subsection{Alpha Zero}
\subsubsection{Introduction}

AlphaZero is a generalized reinforcement learning framework that extends the methodology of AlphaGo Zero beyond the domain of Go. It employs the same core components—self-play reinforcement learning, deep neural networks with policy and value heads, and Monte Carlo Tree Search (MCTS)—but is designed to operate across Shogi, Chess, and Go. With only minimal adjustments to the input and output representations to account for game-specific rules, AlphaZero achieved superhuman performance in Go, Chess, and Shogi, thereby demonstrating the domain-independence and general applicability of the approach.

\subsubsection{Key Innovations}
AlphaZero introduced several advancements over AlphaGo Zero, 
extending its applicability and improving efficiency:

\begin{enumerate}
    \item Generalization Across Games: Unlike AlphaGo Zero, 
    which was designed exclusively for Go, AlphaZero applied the 
    same reinforcement learning framework to \textit{Go, Chess, and Shogi}. 
    With only minimal modifications to the input and output representations 
    to encode different rules and move sets, AlphaZero demonstrated 
    the domain-independence of the algorithm.
    
    \item Game-Agnostic Training Procedure: AlphaZero 
    established the training process as a fully general loop applicable 
    to multiple strategy games. Self-play guided by MCTS generated 
    training examples $(s_t, \pi_t, z)$, and the neural network was 
    updated using the same loss function as in AlphaGo Zero. 
    No game-specific heuristics or specialized adjustments were required, 
    confirming the general-purpose nature of the approach.
    
    \item Improved Efficiency: AlphaZero achieved strong 
    performance with approximately 800 MCTS simulations per move, 
    compared to the $\sim$1,600 simulations typically used by AlphaGo Zero. 
    This reduction highlights the efficiency of its policy--value network 
    in guiding search and reduced the computational cost of training 
    and inference.
    
    \item Cross-Domain Superiority: Beyond outperforming 
    AlphaGo Zero in Go, AlphaZero achieved superhuman results in Chess 
    and Shogi, decisively defeating \textit{Stockfish} and \textit{Elmo}, 
    the strongest domain-specific engines at the time. These results 
    established AlphaZero as the first general reinforcement learning 
    system to surpass highly optimized, handcrafted programs across 
    multiple complex strategy games.
\end{enumerate}

\subsubsection{Performance Benchmarks}

\subsubsubsection{Chess}
AlphaZero was evaluated against Stockfish, the strongest chess engine at the time. After only 24 hours of training from random play, AlphaZero achieved a dominant performance. \\
In a 100-game match under one-minute per move time controls, it recorded the following results: \\
\begin{enumerate}
    \item 28 wins, 72 draws, and 0 losses as White
    \item 25 wins, 73 draws, and 2 losses as Black
\end{enumerate}
These results demonstrated that AlphaZero could surpass a hand-crafted, search-intensive engine within a single day of training. \\

\subsubsubsection{Shogi}
In Shogi, AlphaZero was trained for 24 hours and then matched against Elmo, the world champion program. In a 100-game match under one-minute per move controls, AlphaZero achieved 90 wins, 8 draws, and only 2 losses, a result that established it as the strongest known Shogi player, human or machine, at the time. \\

\subsubsubsection{Go}

AlphaZero was compared directly with AlphaGo Zero, its predecessor and the strongest Go-playing system prior to this work. Despite the short training duration of 24 hours, AlphaZero won 60 games to 40 in a 100-game match. This result demonstrated that AlphaZero not only generalized across domains but also improved upon the performance of domain-specialized systems.





\subsection{MuZero}
\input{sections/7MuZero/MuZero.tex}

\section{Real-World Applications}

\label{sec:applications}

The true value of the deep reinforcement learning paradigm pioneered by the DeepMind models is demonstrated by its successful translation into high-impact real-world applications. These models excel in domains that can be framed as sequential decision-making problems with clear objectives, often discovering novel strategies that surpass human-designed solutions. The following case studies illustrate how algorithms developed for games are now driving innovation across mathematics, computer science, systems engineering, and biology.

\subsection{Matrix Multiplication: Alpha Tensor}
\label{subsec:alphatensor}

A prime example of this translation is \textbf{Alpha Tensor}, a deep RL model that adapts the AlphaZero algorithm to the fundamental problem of discovering efficient algorithms for matrix multiplication \cite{fawzi2022discovering}. Alpha Tensor frames the problem as a single-player game, the \textit{Tensor Game}. The state of the game is a three-dimensional tensor, and actions correspond to updating this tensor. The model is rewarded for finding a path that factorizes the initial matrix multiplication tensor into the zero tensor in the fewest steps, corresponding to the lowest \textit{rank} of the multiplication algorithm.

Searching a space of predefined factor entries, the model discovers the optimal multiplication algorithm from scratch. Through this process, Alpha Tensor discovered algorithms that matched or surpassed the best human-developed ones. A notable achievement was for $4 \times 4$ matrices, where it discovered a rank-47 algorithm, an improvement over the long-standing best-known rank-49 (Strassen's) algorithm. Beyond finding a single optimal solution, Alpha Tensor generated a vast database of distinct, non-equivalent multiplication algorithms, providing a valuable resource for further mathematical research. Furthermore, the model's reward function can be adjusted to optimize for specific hardware capabilities, demonstrating its flexibility in generating hardware-aware algorithms that minimize latency or energy consumption.

\subsection{Sorting Algorithms: AlphaDev}
\label{subsec:alphadev}

Sorting is one of the most common subroutines in programming, and as the demand for computation increases, the use of such fundamental algorithms grows. As such, optimizing them is critical for computational efficiency. \textbf{AlphaDev} adapts the AlphaZero algorithm to this task by modeling the sequence of low-level CPU instructions (assembly code) for a sorting function as a single-player game, the \textit{AssemblyGame} \cite{alphadev2023}. The goal of the game is to find a correct program that minimizes latency (a lower ``score'' is better).

The model searches the game space for the shortest, fastest, and correct algorithm. For fixed-length sorting (e.g., sorting lists of length 3, 4, and 5), it discovered new algorithms that outperformed the best-known human-designed solutions in the standard C++ library. Even for the more complex variable-length sort, AlphaDev generated enhancements. Notably, the improved sorting libraries discovered by AlphaDev have been integrated into the standard C++ \texttt{libc++} library, providing efficiency gains for millions of developers and applications worldwide.

\subsection{Compression Optimization: MuZero RC}
\label{subsec:muzerorc}

A notable implementation of MuZero has been in collaboration with YouTube, where it was used to optimize video compression within the open-source VP9 codec. This implementation, called \textbf{MuZero Rate-Controller (MuZero-RC)}, adapted MuZero's ability to predict and plan to the complex, practical task of video streaming \cite{muzero_real_world_2022}.

By optimizing the encoding process, MuZero-RC achieved an average bitrate reduction of 4\% without degrading video quality. This improvement directly impacts the efficiency of video streaming services such as YouTube, leading to faster loading times and reduced data usage for users. This application exemplifies how the model-based reinforcement learning principles behind MuZero can address practical real-world challenges outside of games, making computer systems more efficient and less resource-intensive.

\subsection{Protein Folding: AlphaFold}
\label{subsec:alphafold}

\textbf{AlphaFold} addresses one of the most challenging problems in biology: predicting the three-dimensional structure of a protein from its amino acid sequence \cite{jumper2021highly}. While primarily a feat of supervised learning, AlphaFold's training was refined using reinforcement learning to improve the accuracy of its predicted protein structures. The model operates on a feedback loop where it is rewarded for generating structures that match known experimental data.

This process of iterative refinement, guided by a reward signal, aligns with the core RL principles explored in this paper. The architecture of AlphaFold includes deep neural networks that analyze both the sequential and spatial relationships between amino acids. By training on extensive datasets of known protein structures, AlphaFold has achieved unprecedented accuracy, often rivaling experimental methods such as X-ray crystallography. This breakthrough has had a transformative impact on biological research and drug discovery.


\section{Future Directions}

\label{sec:future}

The Alpha series demonstrated a remarkable capacity for mastering environments through self-play, a feat extended by MuZero, which learned an internal model without a simulator. Despite their success, these models are fundamentally limited to single-player, deterministic games. The next critical step in their evolution is to expand their scope to more complex and realistic settings. Furthermore, the core environment discovery algorithm must be enhanced to operate effectively in wider state and action spaces. This section discusses pioneering advancements in these directions, including multi-agent generalization, algorithmic improvements for exploration, and handling stochasticity.

\subsection{Model-based Multi-Agent Reinforcement Learning}
\label{subsec:multiagent}

Traditional RL models, including the Alpha series, focus on optimizing a strategy in a stationary environment. However, many real-world applications require collaboration or competition between multiple agents. Deploying multiple agents introduces fundamental challenges: the environment becomes non-stationary from any single agent's perspective, partial observability is common, and the problems of coordination, reward assignment, and scalability become paramount. These challenges make MARL significantly more complex and sample inefficient.

A key future direction is to extend the model-based planning principles of MuZero to these multi-agent settings. Model-based methods have demonstrated higher sample efficiency in complex single-agent environments, a trait that is critically needed to tackle the exponential complexity of MARL \cite{wang2023model}. The goal is to develop agents that can not only predict environment dynamics but also model the strategies of other agents, enabling sophisticated planning in competitive or cooperative spaces.

This remains a formidable open challenge. Projects like AlphaStar \cite{vinyals2019alphastar} show that superhuman performance in complex multi-agent games is possible, but they rely on immense computational resources and specialized architectures. Future research must focus on making such planning efficient and generalizable. For instance, recent work explores using deep networks to model diverse agent behaviors and improve coordination, as seen in approaches that learn diverse Q-vectors \cite{luo2023multi} or use centralized training for decentralized execution, pushing the boundaries of what is possible in cooperative tasks like multi-robot coordination.

\subsection{Algorithmic Enhancements for Exploration and Efficiency}
\label{subsec:algorithmic}

A parallel direction for advancement lies in enhancing the core algorithms themselves to improve their sample efficiency, exploration capabilities, and computational footprint. The monumental computational cost of training models like MuZero from scratch remains a significant barrier to wider application and experimentation.

The \textbf{MiniZero} framework \cite{wu2023minizero} provides a comparative analysis of algorithms like AlphaZero and MuZero, including variants that incorporate stochastic planning techniques. A key innovation in this space is the integration of \textbf{Gumbel noise} into the Monte Carlo Tree Search (MCTS) process \cite{danihelka2022policy}. This approach replaces deterministic action selection with a probabilistic one, leading to more robust exploration and the discovery of novel strategies that might be missed by the standard upper confidence bound heuristic. By improving the planning process itself, these methods aim to achieve stronger performance with fewer simulations and reduced overall computation, making the powerful model-based RL paradigm more accessible and applicable to problems with vast state spaces.

\subsection{Robustness in Stochastic and Partially Observable Environments}
\label{subsec:stochastic}

Finally, a major frontier is equipping these models to handle the inherent uncertainty and randomness of real-world environments. The successes of the Alpha series and MuZero have largely been confined to deterministic and fully observable settings, which are rare outside of perfect simulators and perfect information games.

Future iterations must develop a robustness to stochastic transitions and rewards. This involves creating agents that can maintain performance even when the outcome of an action is not guaranteed or when the reward signal is noisy. Furthermore, models need to effectively address partial observability, where the agent does not have access to the full state of the environment. This requires moving beyond the Markov assumption and learning to maintain and update a belief state over time. Promising research directions include blending the planning strengths of MuZero with techniques from robust reinforcement learning \cite{oikarinen2023robust} and leveraging recurrent architectures to better infer the true state of a partially observed world. Achieving this would represent a final crucial step in transitioning these game-playing algorithms into robust decision-making engines for real-world applications.


\section*{Conclusion}
\section*{Conclusion}

Games, as an environment for reinforcement learning, have proven to be very impactful as sandboxes. Their modular nature enables experimentation across different scenarios, ranging from deterministic board games to visually complex and endless Atari games. 

Google’s DeepMind utilized this modularity by developing and enhancing their models incrementally. The progression started with \textbf{AlphaGo}, which relied on human gameplay and explicit knowledge of game rules. This was followed by \textbf{AlphaGo Zero}, which removed the need for human gameplay data, and \textbf{AlphaZero}, which generalized the approach to multiple board games. Finally, \textbf{MuZero} eliminated the requirement for prior knowledge of game rules entirely, achieving breakthrough results in tens of games and surpassing its predecessors.

These advancements have translated into real-world applications, such as \textbf{MuZero’s optimization of YouTube's compression algorithm}, which was already highly optimized using traditional techniques. Similarly, \textbf{AlphaFold}, while inspired by reinforcement learning principles like those in AlphaZero, relies primarily on supervised learning to model complex proteins. 

While these achievements are impressive, especially given their roots in models trained to play simple games, they remain limited in scope. Challenges such as high training costs, scalability, and performance in stochastic environments persist. 

Firstly, these models are \textbf{expensive to train}, even in environments with limited action and state spaces, and this cost only increases in more complex scenarios. Secondly, \textbf{scalability} is another challenge, as many real-world applications involve actions that are not mutually exclusive. This makes techniques like Monte Carlo Tree Search (MCTS) exponentially more expensive. Lastly, these models, while performing well in deterministic settings, may face difficulties when applied to \textbf{stochastic environments}, affecting both training and inference. 

However, ongoing research in areas such as \textbf{model-based reinforcement learning} and \textbf{hierarchical reinforcement learning} provides hope for addressing these limitations, potentially expanding the applicability of these methods to more complex real-world scenarios.




\begin{thebibliography}{00}

% --- Introduction ---
    \bibitem{skinner1938} B. F. Skinner, \textit{The Behavior of Organisms: An Experimental Analysis}. Appleton-Century, 1938.
    \bibitem{georgios2018} N. Y. Georgios and T. Julian, Artificial Intelligence and Games. New York: Springer, 2018.
    \bibitem{samuel1959} A. L. Samuel, “Some studies in machine learning using the game of checkers,” \textit{IBM J. Res. Dev.}, vol. 3, no. 3, pp. 210–229, 1959.
    \bibitem{minsky1961} M. Minsky, “Steps Toward Artificial Intelligence,” \textit{Proc. IRE}, vol. 49, no. 1, pp. 8–30, 1961.
    \bibitem{kaelbling1996} L. P. Kaelbling, M. L. Littman, and A. W. Moore,“Reinforcement learning: A survey,” \textit{Journal of Artificial Intelligence Research}, vol. 4, pp. 237–285, 1996.
    \bibitem{suttonbarto2018} R. S. Sutton and A. G. Barto,\textit{Reinforcement Learning: An Introduction}, 2nd ed., MIT Press, 2018.
    \bibitem{bellemare2013} M. G. Bellemare, Y. Naddaf, J. Veness, and M. Bowling,“The Arcade Learning Environment: An evaluation platform for general agents,”\textit{Journal of Artificial Intelligence Research}, vol. 47, pp. 253–279, 2013.
    \bibitem{mnih2015} V. Mnih et al.,“Human-level control through deep reinforcement learning,” \textit{Nature}, vol. 518, no. 7540, pp. 529–533, 2015.
    \bibitem{puterman1994} M. L. Puterman, \textit{Markov Decision Processes: Discrete Stochastic Dynamic Programming}. Wiley, 1994.
    \bibitem{howard1960} R. A. Howard, \textit{Dynamic Programming and Markov Processes}. Wiley, 1960.
    \bibitem{bellman1957} R. E. Bellman, \textit{Dynamic Programming}. Princeton University Press, 1957.
    \bibitem{barto1983} A. G. Barto, R. S. Sutton, and C. W. Anderson, “Neuronlike adaptive elements that can solve difficult learning control problems,” \textit{IEEE Trans. Syst., Man, Cybern.}, vol. 13, no. 5, pp. 834–846, 1983.
    \bibitem{sutton1988} R. S. Sutton, “Learning to predict by the methods of temporal differences,” \textit{Machine Learning}, vol. 3, no. 1, pp. 9–44, 1988.
    \bibitem{watkins1992} C.J.C.H.Watkins, P. Dayan, Q-learning. Mach Learn 8, 279–292 (1992).
    \bibitem{dayan1992} P. Dayan, “The convergence of TD($\lambda$) for general $\lambda$,” \textit{Machine Learning}, vol. 8, no. 3–4, pp. 341–362, 1992.
    \bibitem{lecun2015} Y. LeCun, Y. Bengio, and G. Hinton, “Deep learning,” \textit{Nature}, vol. 521, pp. 436–444, 2015.
    \bibitem{goodfellow2016} I. Goodfellow, Y. Bengio, and A. Courville, \textit{Deep Learning}. MIT Press, 2016.
    \bibitem{I2} N. Justesen, P. Bontrager, J. Togelius, S. Risi, (2019). Deep learning for video game playing. arXiv.
    \bibitem{tsitsiklis1997} J. N. Tsitsiklis and B. Van Roy, “An analysis of temporal-difference learning with function approximation,” \textit{IEEE Trans. Autom. Control}, vol. 42, no. 5, pp. 674–690, 1997.
    \bibitem{williams1992} R. J. Williams, “Simple statistical gradient-following algorithms for connectionist reinforcement learning,”
    \bibitem{I3}  V. Mnih, K. Kavukcuoglu, D. Silver, A. Graves, I. Antonoglou, D. Wierstra, M. Riedmiller, (2013). Playing Atari with deep reinforcement learning. arXiv.
    \bibitem{I4}  A. Graves, G. Wayne, I. Danihelka, (2014). Neural Turing Machines. arXiv.
    \bibitem{I6}  DeepMind, (2015, February 12), Deep reinforcement learning.
    \bibitem{watkins1989} C. J. C. H. Watkins, \textit{Learning from Delayed Rewards}, PhD thesis, Univ. of Cambridge, 1989.
    \bibitem{I7}  T. Schaul, J. Quan, I. Antonoglou, D. Silver, (2015). Prioritized Experience Replay. arXiv.
    \bibitem{lin1992} L.-J. Lin, “Self-improving reactive agents based on reinforcement learning, planning and teaching,” \textit{Machine Learning}, vol. 8, no. 3–4, pp. 293–321, 1992.
    \bibitem{I8}  V. Mnih, A. P. Badia, M. Mirza, A. Graves, T. Lillicrap, T. Harley, D. Silver, K. Kavukcuoglu, (2016). Asynchronous Methods for Deep Reinforcement Learning. arXiv.
    \bibitem{konda2000} V. R. Konda and J. N. Tsitsiklis, “Actor-Critic Algorithms,” in \textit{Advances in Neural Information Processing Systems}, vol. 12, 2000.
    \bibitem{Silver2016} D. Silver et al., “Mastering the game of Go with deep neural networks and tree search,” \textit{Nature}, vol. 529, pp. 484–489, 2016.
    \bibitem{I12} K. Shao, Z. Tang, Y. Zhu, N. Li, D. Zhao, (2019). A survey of deep reinforcement learning in video games. arXiv.

    % --- Related Work order ---
    \bibitem{I9}  A. Kailash, P. D. Marc, B. Miles, and A. B. Anil, (2017). Deep Reinforcement Learning: A Brief Survey. IEEE Signal Processing Magazine, vol. 34, pp. 26–38, 2017. arXiv.
    \bibitem{I10} D. Zhao,  K. Shao, Y. Zhu, D. Li, Y. Chen, H. Wang, D. Liu, T. Zhou, and C. Wang, “Review of deep reinforcement learning and discussions on the development of computer Go,” Control Theory and Applications, vol. 33, no. 6, pp. 701–717, 2016 arXiv.
    \bibitem{I11} Z. Tang, K. Shao, D. Zhao, and Y. Zhu, “Recent progress of deep reinforcement learning: from AlphaGo to AlphaGo Zero,” Control Theory and Applications, vol. 34, no. 12, pp. 1529–1546, 2017.
    \bibitem{hessel2018} M. Hessel et al., “Rainbow: Combining improvements in deep reinforcement learning,”in \textit{Proc. AAAI}, 2018, pp. 3215–3222.
    \bibitem{lillicrap2015} T. P. Lillicrap et al.,“Continuous control with deep reinforcement learning,” arXiv:1509.02971, 2015.
    \bibitem{schulman2015} J. Schulman, S. Levine, P. Abbeel, M. Jordan, and P. Moritz,“Trust region policy optimization,” in \textit{Proc. ICML}, 2015.
    \bibitem{schulman2017} J. Schulman, F. Wolski, P. Dhariwal, A. Radford, and O. Klimov,“Proximal Policy Optimization Algorithms,” arXiv:1707.06347, 2017.
    \bibitem{espeholt2018} L. Espeholt et al.,“IMPALA: Scalable distributed deep-RL with importance weighted actor-learner architectures,” in \textit{Proc. ICML}, 2018.
    \bibitem{lowe2017} R. Lowe, Y. Wu, A. Tamar, J. Harb, P. Abbeel, and I. Mordatch, “Multi-agent actor-critic for mixed cooperative-competitive environments,”in \textit{Proc. NeurIPS}, 2017.
    \bibitem{AD2} K. Zhang, Z. Yang, and T. Başar,“Multi-agent reinforcement learning: A selective overview of theories and algorithms,”\textit{Handbook of Reinforcement Learning and Control}, Springer, 2019.
    \bibitem{vinyals2019} O. Vinyals et al.,“Grandmaster level in StarCraft II using multi-agent reinforcement learning,”\textit{Nature}, vol. 575, pp. 350–354, 2019.
    \bibitem{jaderberg2019} M. Jaderberg et al.,“Human-level performance in 3D multiplayer games with population-based reinforcement learning,”\textit{Science}, vol. 364, no. 6443, pp. 859–865, 2019.
    \bibitem{bellemare2016} M. Bellemare, S. Srinivasan, G. Ostrovski, T. Schaul, D. Saxton, and R. Munos,“Unifying count-based exploration and intrinsic motivation,”in \textit{Proc. NeurIPS}, 2016.
    \bibitem{pathak2017} D. Pathak, P. Agrawal, A. Efros, and T. Darrell,“Curiosity-driven exploration by self-supervised prediction,”in \textit{Proc. CVPR}, 2017.
    \bibitem{zhang2018generalization} C. Zhang, O. Vinyals, R. Munos, and S. Bengio,“A study on overfitting in deep reinforcement learning,” arXiv:1804.06893, 2018.
    \bibitem{kirk2023} R. Kirk, C. Zhang, J. Parker-Holder, Y. Lu, A. Gleave,and J. Foerster, “Survey of generalisation in deep reinforcement learning,”\textit{Artificial Intelligence Review}, vol. 56, pp. 3003–3037, 2023.
    \bibitem{zhang2019multi} K. Zhang, Z. Yang, and T. Başar,“Multi-agent reinforcement learning: A selective overview of theories and algorithms,”\textit{Handbook of Reinforcement Learning and Control}, Springer, 2019.
    \bibitem{ecoffet2021} A. Ecoffet, J. Huizinga, J. Lehman, K. Stanley, and J. Clune,“First return, then explore,” \textit{Nature}, vol. 590, pp. 580–586, 2021.
    \bibitem{taylor2009} M. E. Taylor and P. Stone,“Transfer learning for reinforcement learning domains: A survey,”\textit{Journal of Machine Learning Research}, vol. 10, pp. 1633–1685, 2009.
    \bibitem{li2018survey} Y. Li, “Deep reinforcement learning: An overview,”\textit{arXiv:1701.07274}, updated 2018.
    \bibitem{francois2018} V. Francois-Lavet, P. Henderson, R. Islam, M. G. Bellemare,J. Pineau, and D. Precup, “An introduction to deep reinforcement learning,”\textit{Foundations and Trends in Machine Learning}, vol. 11, no. 3–4,pp. 219–354, 2018.


    %-------Background-------
    \bibitem{bg1} L. Thorndike and D. Bruce, Animal Intelligence. Routledge, 2017.
    \bibitem{bg2} R. S. Sutton and A. Barto, Reinforcement learning : an introduction. Cambridge, Ma ; London: The Mit Press, 2018.
    \bibitem{bg3} A. Kumar Shakya, G. Pillai, and S. Chakrabarty, “Reinforcement Learning Algorithms: A brief survey,” Expert Systems with Applications, vol. 231, p. 120495, May 2023
    \bibitem{bg4} Mnih, Volodymyr, et al. “Human-Level Control through Deep Reinforcement Learning.” Nature, vol. 518, no. 7540, Feb. 2015, pp. 529–533.
  %-------AlphaGo-------
    \bibitem{Silver2016} D. Silver et al., “Mastering the game of Go with deep neural networks and tree search,” \textit{Nature}, vol. 529, no. 7587, pp. 484–489, Jan. 2016, doi: https://doi.org/10.1038/nature16961.
    %-------AlphaGoZero-------
    \bibitem{agz1} Silver, David, Julian Schrittwieser, Karen Simonyan, Ioannis Antonoglou, Aja Huang, Arthur Guez, Thomas Hubert, et al. 2017. “Mastering the Game of Go without Human Knowledge.” Nature 550 (7676): 354–59. https://doi.org/10.1038/nature24270.

    % ---------------- AlphaZero --------------
    \bibitem{az1} David Silver et al. ,A general reinforcement learning algorithm that masters chess, shogi, and Go through self-play. Science 362,1140-1144(2018). doi:10.1126/science.aar6404

    %-------Muzero-------
    \bibitem{mz1} J. Schrittwieser et al., “Mastering Atari, go, chess and shogi by planning with a learned model,” Nature, vol. 588, no. 7839, pp. 604–609, Dec. 2020. doi:10.1038/s41586-020-03051-4 
    %-------Advancememts-------
    \bibitem{fawzi2022discovering}
    A. Fawzi et al., ``Discovering faster matrix multiplication algorithms with reinforcement learning,'' \textit{Nature}, vol. 610, no. 7930, pp. 47--53, Oct. 2022, doi: 10.1038/s41586-022-05172-4.
    
    \bibitem{alphadev2023}
    D. J. Mankowitz et al., ``Faster sorting algorithms discovered using deep reinforcement learning,'' \textit{Nature}, vol. 618, no. 7964, pp. 257–263, Jun. 2023, doi: 10.1038/s41586-023-06004-9.
    
    \bibitem{muzero_real_world_2022}
    DeepMind, ``MuZero's first step from research into the real world,'' DeepMind, Feb. 11, 2022. [Online]. Available: \url{https://www.deepmind.com/blog/muzeros-first-step-from-research-into-the-real-world}. [Accessed: Oct. 1, 2023].
    
    \bibitem{jumper2021highly}
    J. Jumper et al., ``Highly accurate protein structure prediction with AlphaFold,'' \textit{Nature}, vol. 596, no. 7873, pp. 583–589, Aug. 2021, doi: 10.1038/s41586-021-03819-2.

    
    %-------Future Directions-------
    \bibitem{wang2023model} X. Wang, Z. Zhang, W. Zhang, ``Model-based Multi-agent Reinforcement Learning: Recent Progress and Prospects,'' \textit{arXiv preprint arXiv:2303.07883}, 2023.

    \bibitem{luo2023multi} Z. Luo, Z. Chen, J. Welsh, ``Multi-agent Reinforcement Learning with Deep Networks for Diverse Q-Vectors,'' \textit{IEEE Transactions on Robotics}, vol. 39, no. 2, 2023.
    
    \bibitem{vinyals2019alphastar} O. Vinyals et al., ``Grandmaster level in StarCraft II using multi-agent reinforcement learning,'' \textit{Nature}, vol. 575, no. 7782, pp. 350–354, 2019.
    
    \bibitem{wu2023minizero} T.-R. Wu et al., ``MiniZero: Comparative Analysis of AlphaZero and MuZero in Go, Othello, and Atari Games,'' \textit{arXiv preprint arXiv:2304.06826}, 2023.
    
    \bibitem{danihelka2022policy} I. Danihelka et al., ``Policy improvement by planning with gumbel,'' in \textit{International Conference on Learning Representations (ICLR)}, 2022.
    
    \bibitem{oikarinen2023robust} T. Oikarinen et al., ``Robust Reinforcement Learning via Adversarial Kernel Approximation,'' in \textit{International Conference on Learning Representations (ICLR)}, 2023.
\bibliographystyle{plain}
\bibliography{references}

\end{thebibliography}
\vspace{12pt}

\end{document}
